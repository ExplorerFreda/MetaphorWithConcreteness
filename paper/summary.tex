\section{Related Work}
\subsection{Linggle}
\par


\section{Experiments}

\subsection{Feature Collection}
To indicate whether an adjective is a central property of a noun, we apply the following features.
\subsubsection{Frequency and Collocation}

\subsubsection{PMI}
Pairwise mutual information (or point mutual information), known as PMI \cite{church1990word, bouma2009normalized}, is a widely accepted measure that illustrates the association between two words. 
The PMI between two words $x$ and $y$ can be computed by 
\begin{equation}
PMI(x, y) = \log_2 \frac{P(x,y)}{P(x)P(y)}
\end{equation}
where $P(x)$ is the probability of word $x$ to appear, 
$P(x,y)$ is the joint probability of $x$ and $y$, namely the probability of the collocation of $x$ and $y$. 
Based on the statistics, it can be estimated by the frequency of word $x$ and $y$,  with the collocation frequency of them. 
In application, it is usually computed by 
\begin{equation}
\begin{aligned}
PMI(x,y) &= \log_2 \frac{C(x,y) \cdot N}{C(x)C(y)}  \\
 &= \log_2{C(x,y)} - \log_2{C(x)} \\ 
 & \quad - \log_2{C(y)} + \log_2 N
\end{aligned}
\end{equation}
where $N$ is the number of unigrams in total, 
$C(x)$ and $C(y)$ represent the frequency of unigram $x$ and $y$, 
and $C(x,y)$ represents the frequency of bigram $(x, y)$.

\subsubsection{Concreteness of words}
\newcite{turney2011literal} proposed a method to compute the concreteness and abstractness of words, and suggested that concreteness of words are strong indicators of metaphors. 
We replicated the experiments for concreteness computation with the method on a large web scale corpus, and got a list with 75,626 words' concreteness.
We also treat concreteness as a feature for central attributes extraction.

\subsubsection{"As ... as" pattern}
There is a consensus that the sentence {\sl Something is as A as a/an N} is reasonable if {\sl A} is a central property of {\sl N}.\cite{ano} 
Thus, the frequency of cooccurrence of an adjective and a noun pair in the pattern of "as adj. as a/an noun." would be a strong feature to detect central properties of nouns. 
We used Linggle API \cite{boisson2013linggle} \footnote{add the queries after the url {\sl http://linggle.com/query/}} to extract such patterns, with the queries like ``as strong as a/an n.''  and ``as adj. as a/an tiger'' , and it would return 50 most frequent matched phrases.



\subsection{Annotation}


\subsection{Future Work}
\begin{itemize}
\item The concept of central property should not be limited in only adjectives. For example, ``have sharp teeth'' should be a central property of tigers or lions, so that something could be compared to them in order to indicate such a property.
\end{itemize}